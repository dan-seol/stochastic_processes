\documentclass[]{article}
\usepackage{lmodern}
\usepackage{amssymb,amsmath}
\usepackage{ifxetex,ifluatex}
\usepackage{fixltx2e} % provides \textsubscript
\ifnum 0\ifxetex 1\fi\ifluatex 1\fi=0 % if pdftex
  \usepackage[T1]{fontenc}
  \usepackage[utf8]{inputenc}
\else % if luatex or xelatex
  \ifxetex
    \usepackage{mathspec}
  \else
    \usepackage{fontspec}
  \fi
  \defaultfontfeatures{Ligatures=TeX,Scale=MatchLowercase}
\fi
% use upquote if available, for straight quotes in verbatim environments
\IfFileExists{upquote.sty}{\usepackage{upquote}}{}
% use microtype if available
\IfFileExists{microtype.sty}{%
\usepackage{microtype}
\UseMicrotypeSet[protrusion]{basicmath} % disable protrusion for tt fonts
}{}
\usepackage[margin=1in]{geometry}
\usepackage{hyperref}
\hypersetup{unicode=true,
            pdftitle={MATH 447 - Assignment 1},
            pdfborder={0 0 0},
            breaklinks=true}
\urlstyle{same}  % don't use monospace font for urls
\usepackage{color}
\usepackage{fancyvrb}
\newcommand{\VerbBar}{|}
\newcommand{\VERB}{\Verb[commandchars=\\\{\}]}
\DefineVerbatimEnvironment{Highlighting}{Verbatim}{commandchars=\\\{\}}
% Add ',fontsize=\small' for more characters per line
\usepackage{framed}
\definecolor{shadecolor}{RGB}{248,248,248}
\newenvironment{Shaded}{\begin{snugshade}}{\end{snugshade}}
\newcommand{\AlertTok}[1]{\textcolor[rgb]{0.94,0.16,0.16}{#1}}
\newcommand{\AnnotationTok}[1]{\textcolor[rgb]{0.56,0.35,0.01}{\textbf{\textit{#1}}}}
\newcommand{\AttributeTok}[1]{\textcolor[rgb]{0.77,0.63,0.00}{#1}}
\newcommand{\BaseNTok}[1]{\textcolor[rgb]{0.00,0.00,0.81}{#1}}
\newcommand{\BuiltInTok}[1]{#1}
\newcommand{\CharTok}[1]{\textcolor[rgb]{0.31,0.60,0.02}{#1}}
\newcommand{\CommentTok}[1]{\textcolor[rgb]{0.56,0.35,0.01}{\textit{#1}}}
\newcommand{\CommentVarTok}[1]{\textcolor[rgb]{0.56,0.35,0.01}{\textbf{\textit{#1}}}}
\newcommand{\ConstantTok}[1]{\textcolor[rgb]{0.00,0.00,0.00}{#1}}
\newcommand{\ControlFlowTok}[1]{\textcolor[rgb]{0.13,0.29,0.53}{\textbf{#1}}}
\newcommand{\DataTypeTok}[1]{\textcolor[rgb]{0.13,0.29,0.53}{#1}}
\newcommand{\DecValTok}[1]{\textcolor[rgb]{0.00,0.00,0.81}{#1}}
\newcommand{\DocumentationTok}[1]{\textcolor[rgb]{0.56,0.35,0.01}{\textbf{\textit{#1}}}}
\newcommand{\ErrorTok}[1]{\textcolor[rgb]{0.64,0.00,0.00}{\textbf{#1}}}
\newcommand{\ExtensionTok}[1]{#1}
\newcommand{\FloatTok}[1]{\textcolor[rgb]{0.00,0.00,0.81}{#1}}
\newcommand{\FunctionTok}[1]{\textcolor[rgb]{0.00,0.00,0.00}{#1}}
\newcommand{\ImportTok}[1]{#1}
\newcommand{\InformationTok}[1]{\textcolor[rgb]{0.56,0.35,0.01}{\textbf{\textit{#1}}}}
\newcommand{\KeywordTok}[1]{\textcolor[rgb]{0.13,0.29,0.53}{\textbf{#1}}}
\newcommand{\NormalTok}[1]{#1}
\newcommand{\OperatorTok}[1]{\textcolor[rgb]{0.81,0.36,0.00}{\textbf{#1}}}
\newcommand{\OtherTok}[1]{\textcolor[rgb]{0.56,0.35,0.01}{#1}}
\newcommand{\PreprocessorTok}[1]{\textcolor[rgb]{0.56,0.35,0.01}{\textit{#1}}}
\newcommand{\RegionMarkerTok}[1]{#1}
\newcommand{\SpecialCharTok}[1]{\textcolor[rgb]{0.00,0.00,0.00}{#1}}
\newcommand{\SpecialStringTok}[1]{\textcolor[rgb]{0.31,0.60,0.02}{#1}}
\newcommand{\StringTok}[1]{\textcolor[rgb]{0.31,0.60,0.02}{#1}}
\newcommand{\VariableTok}[1]{\textcolor[rgb]{0.00,0.00,0.00}{#1}}
\newcommand{\VerbatimStringTok}[1]{\textcolor[rgb]{0.31,0.60,0.02}{#1}}
\newcommand{\WarningTok}[1]{\textcolor[rgb]{0.56,0.35,0.01}{\textbf{\textit{#1}}}}
\usepackage{graphicx,grffile}
\makeatletter
\def\maxwidth{\ifdim\Gin@nat@width>\linewidth\linewidth\else\Gin@nat@width\fi}
\def\maxheight{\ifdim\Gin@nat@height>\textheight\textheight\else\Gin@nat@height\fi}
\makeatother
% Scale images if necessary, so that they will not overflow the page
% margins by default, and it is still possible to overwrite the defaults
% using explicit options in \includegraphics[width, height, ...]{}
\setkeys{Gin}{width=\maxwidth,height=\maxheight,keepaspectratio}
\IfFileExists{parskip.sty}{%
\usepackage{parskip}
}{% else
\setlength{\parindent}{0pt}
\setlength{\parskip}{6pt plus 2pt minus 1pt}
}
\setlength{\emergencystretch}{3em}  % prevent overfull lines
\providecommand{\tightlist}{%
  \setlength{\itemsep}{0pt}\setlength{\parskip}{0pt}}
\setcounter{secnumdepth}{0}
% Redefines (sub)paragraphs to behave more like sections
\ifx\paragraph\undefined\else
\let\oldparagraph\paragraph
\renewcommand{\paragraph}[1]{\oldparagraph{#1}\mbox{}}
\fi
\ifx\subparagraph\undefined\else
\let\oldsubparagraph\subparagraph
\renewcommand{\subparagraph}[1]{\oldsubparagraph{#1}\mbox{}}
\fi

%%% Use protect on footnotes to avoid problems with footnotes in titles
\let\rmarkdownfootnote\footnote%
\def\footnote{\protect\rmarkdownfootnote}

%%% Change title format to be more compact
\usepackage{titling}

% Create subtitle command for use in maketitle
\newcommand{\subtitle}[1]{
  \posttitle{
    \begin{center}\large#1\end{center}
    }
}

\setlength{\droptitle}{-2em}

  \title{MATH 447 - Assignment 1}
    \pretitle{\vspace{\droptitle}\centering\huge}
  \posttitle{\par}
    \author{}
    \preauthor{}\postauthor{}
    \date{}
    \predate{}\postdate{}
  
\usepackage{blkarray}

\begin{document}
\maketitle

Dan Yunheum Seol 260677676

\hypertarget{section}{%
\subsection{1.17}\label{section}}

We have \[
X \sim Pois(\lambda) : \lambda = 3
\] and we need to obtain \[
E[X|X > 2]
\]

We use the Law of Iterated Expectation \[
\lambda = E[X] = E[E[X|X>2]] = E[X|X>2]P(X >2) + E[X| X \leq 2] P(X \leq 2)
\] It follows that \[
E[X|X>2] = \frac{E[X]-E[X|X \leq 2]}{P(X \leq 2)} = \frac{\lambda - \sum_{x=0}^2 x  \frac{e^{-\lambda} \lambda^x}{x!} }{\sum_{x=0}^2   \frac{e^{-\lambda} \lambda^x}{x!}} = \frac{3 - (0 + 3e^{-3} + 9e^{-3})}{e^{-3} + 3*e^{-3}+\frac{9}{2}e^{-3}}
\] R gives the result of

\begin{Shaded}
\begin{Highlighting}[]
\KeywordTok{round}\NormalTok{((}\DecValTok{3-3}\OperatorTok{*}\KeywordTok{exp}\NormalTok{(}\OperatorTok{-}\DecValTok{3}\NormalTok{)}\OperatorTok{*}\NormalTok{(}\DecValTok{1}\OperatorTok{+}\DecValTok{3}\NormalTok{))}\OperatorTok{/}\NormalTok{(}\DecValTok{1}\OperatorTok{-}\NormalTok{(}\KeywordTok{exp}\NormalTok{(}\OperatorTok{-}\DecValTok{3}\NormalTok{)}\OperatorTok{+}\DecValTok{3}\OperatorTok{*}\KeywordTok{exp}\NormalTok{(}\OperatorTok{-}\DecValTok{3}\NormalTok{)}\OperatorTok{+}\FloatTok{4.5}\OperatorTok{*}\KeywordTok{exp}\NormalTok{(}\OperatorTok{-}\DecValTok{3}\NormalTok{))), }\DecValTok{5}\NormalTok{)}
\end{Highlighting}
\end{Shaded}

\begin{verbatim}
## [1] 4.16525
\end{verbatim}

\(\square\)

\hypertarget{section-1}{%
\subsection{1. 28}\label{section-1}}

We have \[
N := \text{The number of accidents};\ N|\Lambda \sim Pois(\Lambda);\ \Lambda \sim U(0,3)
\] Then by the Laws of Iterated Expectation and Iterated Variance, \[
E[N] = E_{\Lambda}[E_{N}[N| \Lambda]] = E_{\Lambda}[\Lambda] = \frac{3-0}{2} = \frac{3}{2}
\]

\[
Var[N] = E[Var[N|\Lambda]] + Var[E[N|\Lambda]] = E[\Lambda] + Var[\Lambda] = \frac{3}{2} + (\int_0^3 \frac{1}{3} \Lambda^2 d \Lambda - \frac{9}{4} ) = \frac{3}{2} + \frac{\Lambda^3}{9}|^3_0 - \frac{9}{4} =
\]

\[
\frac{3}{2} + (3 - \frac{9}{4}) =  \frac{6}{4} + \frac{3}{4} = \frac{9}{4}
\]

\hypertarget{section-2}{%
\subsection{2.2}\label{section-2}}

We have \((\{X_t\}_{t \in \mathbb{N}_0} , P, \vec{\alpha})\) as our
Markov chain with transition probability matrix \(P\), and initial
distribution \(\vec{\alpha}\) We also have that (the blkarray package
had some errors.)

\$\$

P =

\begin{array}{c|c|c|c}
 & \text{State 1} & \text{State 2} & \text{State 3} \\ 

\text{State 1} & 0.575 & 0.118 & 0.172 \\

\text{State 2} & 0.453 & 0.243 & 0.148 \\
\text{State 3} & 0.104 & 0.343 & 0.367 \\
\end{array}

\$\$

\[
\vec{\alpha} = (1/2, 0, 1/2)
\]

\#\#\#(a)

\[
P(X_2 =1 | X_1 = 3) = P(X_1=1 | X_0 = 3) = P_{31} = \frac{1}{3}
\]

\#\#\#(b) \[
P(X_2 =1,  X_1 = 3) = (\vec{\alpha}P)_3 P_{31} = \frac{5}{12} * \frac{1}{3} = \frac{5}{36}
\]

\begin{Shaded}
\begin{Highlighting}[]
\NormalTok{P1 =}\StringTok{ }\KeywordTok{matrix}\NormalTok{(}\KeywordTok{c}\NormalTok{(}\DecValTok{0}\NormalTok{, }\FloatTok{0.5}\NormalTok{, }\FloatTok{0.5}\NormalTok{, }\DecValTok{1}\NormalTok{, }\DecValTok{0}\NormalTok{, }\DecValTok{0}\NormalTok{, }\DecValTok{1}\OperatorTok{/}\DecValTok{3}\NormalTok{, }\DecValTok{1}\OperatorTok{/}\DecValTok{3}\NormalTok{, }\DecValTok{1}\OperatorTok{/}\DecValTok{3}\NormalTok{), }\DataTypeTok{nrow=}\DecValTok{3}\NormalTok{, }\DataTypeTok{ncol=}\DecValTok{3}\NormalTok{, }\DataTypeTok{byrow=}\OtherTok{TRUE}\NormalTok{)}
\NormalTok{P1}
\end{Highlighting}
\end{Shaded}

\begin{verbatim}
##           [,1]      [,2]      [,3]
## [1,] 0.0000000 0.5000000 0.5000000
## [2,] 1.0000000 0.0000000 0.0000000
## [3,] 0.3333333 0.3333333 0.3333333
\end{verbatim}

\begin{Shaded}
\begin{Highlighting}[]
\NormalTok{alpha =}\StringTok{ }\KeywordTok{c}\NormalTok{(}\FloatTok{0.5}\NormalTok{, }\DecValTok{0}\NormalTok{, }\FloatTok{0.5}\NormalTok{)}
\NormalTok{alpha}
\end{Highlighting}
\end{Shaded}

\begin{verbatim}
## [1] 0.5 0.0 0.5
\end{verbatim}

\begin{Shaded}
\begin{Highlighting}[]
\NormalTok{alpha }\OperatorTok\StringTok{ }\NormalTok{P1}
\end{Highlighting}
\end{Shaded}

\begin{verbatim}
##           [,1]      [,2]      [,3]
## [1,] 0.1666667 0.4166667 0.4166667
\end{verbatim}

\begin{Shaded}
\begin{Highlighting}[]
\NormalTok{((alpha }\OperatorTok\StringTok{ }\NormalTok{P1)[}\DecValTok{3}\NormalTok{]}\OperatorTok{*}\NormalTok{P1[}\DecValTok{3}\NormalTok{][}\DecValTok{1}\NormalTok{])}
\end{Highlighting}
\end{Shaded}

\begin{verbatim}
## [1] 0.1388889
\end{verbatim}

\begin{Shaded}
\begin{Highlighting}[]
\DecValTok{5}\OperatorTok{/}\DecValTok{36}
\end{Highlighting}
\end{Shaded}

\begin{verbatim}
## [1] 0.1388889
\end{verbatim}

\#\#\#(c)

\[
P(X_1=3|X_2 = 1) = \frac{P(X_1=3, X_2 =1)}{P(X_2=1)} = \frac{1}{P(X_2=1)} \frac{5}{36}
\] \[
P(X_2=1) = (\vec{\alpha}P^2)_1 = \frac{60}{108}
\]

\hypertarget{code-for-computing-pn-from-textbook}{%
\subsubsection{\texorpdfstring{Code for computing \(P^n\) from
textbook}{Code for computing P\^{}n from textbook}}\label{code-for-computing-pn-from-textbook}}

\begin{Shaded}
\begin{Highlighting}[]
\CommentTok{##### Matrix powers ###############################}
\CommentTok{# matrixpower(mat,k) mat^k}
\CommentTok{#}
\NormalTok{matrixpower =}\StringTok{ }\ControlFlowTok{function}\NormalTok{(X, n)\{}
  
  \ControlFlowTok{if}\NormalTok{(}\KeywordTok{dim}\NormalTok{(X)[}\DecValTok{1}\NormalTok{] }\OperatorTok{!=}\StringTok{ }\KeywordTok{dim}\NormalTok{(X)[}\DecValTok{2}\NormalTok{])\{}
     \KeywordTok{throw}\NormalTok{(}\StringTok{"Dimensions of the matrix do not match: "}\NormalTok{, }\KeywordTok{dim}\NormalTok{(X))}
\NormalTok{  \}}
  \ControlFlowTok{if}\NormalTok{ (n}\OperatorTok{==}\DecValTok{0}\NormalTok{)\{}\KeywordTok{return}\NormalTok{(}\KeywordTok{diag}\NormalTok{(}\KeywordTok{dim}\NormalTok{(X)[}\DecValTok{1}\NormalTok{]))\}}
  \ControlFlowTok{else} \ControlFlowTok{if}\NormalTok{(n }\OperatorTok{>}\DecValTok{0}\NormalTok{) \{}\KeywordTok{return}\NormalTok{(X }\OperatorTok\StringTok{ }\KeywordTok{matrixpower}\NormalTok{(X, (n}\DecValTok{-1}\NormalTok{)))\}}
  \ControlFlowTok{else}\NormalTok{ \{}\KeywordTok{return}\NormalTok{(}\KeywordTok{matrixpower}\NormalTok{(}\KeywordTok{solve}\NormalTok{(X), }\OperatorTok{-}\NormalTok{n))\}}
\NormalTok{\}}
\end{Highlighting}
\end{Shaded}

\begin{Shaded}
\begin{Highlighting}[]
\NormalTok{P1sq =}\StringTok{ }\KeywordTok{matrixpower}\NormalTok{(P1, }\DecValTok{2}\NormalTok{)}
\NormalTok{P1sq}
\end{Highlighting}
\end{Shaded}

\begin{verbatim}
##           [,1]      [,2]      [,3]
## [1,] 0.6666667 0.1666667 0.1666667
## [2,] 0.0000000 0.5000000 0.5000000
## [3,] 0.4444444 0.2777778 0.2777778
\end{verbatim}

\begin{Shaded}
\begin{Highlighting}[]
\NormalTok{alphaP1sq}\FloatTok{.1}\NormalTok{ =}\StringTok{ }\NormalTok{(alpha }\OperatorTok\StringTok{ }\NormalTok{P1sq)[}\DecValTok{1}\NormalTok{]}
\NormalTok{alphaP1sq}\FloatTok{.1}
\end{Highlighting}
\end{Shaded}

\begin{verbatim}
## [1] 0.5555556
\end{verbatim}

\begin{Shaded}
\begin{Highlighting}[]
\DecValTok{60}\OperatorTok{/}\DecValTok{108}
\end{Highlighting}
\end{Shaded}

\begin{verbatim}
## [1] 0.5555556
\end{verbatim}

\[
P(X_1=3|X_2 = 1) = \frac{108}{60} * \frac{5}{36} =  \frac{1}{4}
\]

\begin{Shaded}
\begin{Highlighting}[]
\NormalTok{(}\DecValTok{5}\OperatorTok{/}\DecValTok{36}\NormalTok{)}\OperatorTok{/}\NormalTok{alphaP1sq}\FloatTok{.1}
\end{Highlighting}
\end{Shaded}

\begin{verbatim}
## [1] 0.25
\end{verbatim}

\#\#\#(d)

\[
P(X_9=1 | X_1 = 3, X_4=1, X_7=2) = P(X_9=1|X_7=2) = P(X_2=1|X_0=2) = P^2_{21} = 0
\] by Markov property.

\begin{Shaded}
\begin{Highlighting}[]
\NormalTok{P1sq[}\DecValTok{2}\NormalTok{][}\DecValTok{1}\NormalTok{]}
\end{Highlighting}
\end{Shaded}

\begin{verbatim}
## [1] 0
\end{verbatim}

\#\#2.7

We have \((\{X_t\}_{t \in \mathbb{N}_0} , P)\), a Markov chain with TPM
P.

We need to show that \(\{Y_t\} = \{X_{3t}\}\) is a Markov chain and show
its TPM.

We show that the property \$P(Y\_\{n+1\}=j \textbar{}Y\_n=i , \ldots{}
Y\_0 = i\_0) = P(Y\_\{n+1\} = j\textbar{} Y\_n = i) \$ for
\(\forall n \in \mathbb{N}_0\) by induction on n.

\#\#\#Base case For n= 0

\[
P(Y_1 = j | Y_{1-1} = i) = P(Y_1 = j | Y_0 = i) =  P(X_3 = j | X_0 = i)  = P^3_{ij}
\]

holds trivially.

\#\#\#Inductive step

Suppose our claim holds for some \(n \in \mathbb{N}_0\) i.e.

\[
\exists n: P(Y_{n+1}=j |Y_n=i , ... Y_0 = i_0) = P(Y_{n+1} = j| Y_n = i)
\]

for \(n+1\) \[
P(Y_{n+2}=j |Y_{n+1}=i , ... Y_0 = i_0) = P(X_{3n+6}=j |X_{3n+3}=i , ... X_0 = i_0) =
\] \[
P(X_{3n+6}=j |X_{3n+3}=i) = P(Y_{n+2}=j |Y_{n+1}=i)
\]

by Markov property of \(\{X_t\}\).

By this we have shown that \(\{Y_t\}\) is a Markov chain. TPM for
\(\{Y_t\}\) will be \(P^3\).

\#\#\#2.13

\$\$ \begin{align}
P = 
\begin{array}{c|c|c|c|c|c|c}
 & \text{abc} & \text{acb} & \text{bac} & \text{bca} & \text{cab}  & \text{cba}  \\ 

\text{abc} & \text{p_a} &  \text{p_c} & \text{p_b} & 0& 0&0 \\

\text{acb}& \text{p_b} & \text{p_a} & 0 & 0  & \text{p_c} & 0  \\
\text{bac} & \text{p_a} & 0 & \text{p_b} & \text{p_c} & 0 & 0 \\
\text{bca} & 0 & 0 & \text{p_a} & \text{p_b} & 0 & \text{p_c}  \\
\text{cab} & 0 & \text{p_a} & 0 & 0 & \text{p_c}  & \text{p_b} \\
\text{cba}  & 0 & 0 & 0 & \text{p_b} & \text{p_a} & \text{p_c}  \\ 
\end{array} 
\end{align} \$\$

where \$\text{p_i} = \$ (probability of book i being returned). If p\_i
= \(\frac{1}{3}\) for \(\forall i\), we will have

\$\$ \begin{align}
P = 
\begin{array}{c|c|c|c|c|c|c}
 & \text{abc} & \text{acb} & \text{bac} & \text{bca} & \text{cab}  & \text{cba}  \\ 

\text{abc} & \frac{1}{3} &   \frac{1}{3} &  \frac{1}{3} & 0& 0&0 \\

\text{acb}&  \frac{1}{3} &  \frac{1}{3} & 0 & 0  &  \frac{1}{3} & 0  \\
\text{bac} &  \frac{1}{3} & 0 & \frac{1}{3} & \frac{1}{3} & 0 & 0 \\
\text{bca} & 0 & 0 & \frac{1}{3} &  \frac{1}{3} & 0 &  \frac{1}{3}  \\
\text{cab} & 0 &  \frac{1}{3} & 0 & 0 &  \frac{1}{3}  &  \frac{1}{3} \\
\text{cba}  & 0 & 0 & 0 &  \frac{1}{3} & \frac{1}{3} &  \frac{1}{3} \\ 
\end{array} 
\end{align} \$\$


\end{document}
