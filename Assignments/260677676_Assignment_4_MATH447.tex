\documentclass[]{article}
\usepackage{lmodern}
\usepackage{amssymb,amsmath}
\usepackage{ifxetex,ifluatex}
\usepackage{fixltx2e} % provides \textsubscript
\ifnum 0\ifxetex 1\fi\ifluatex 1\fi=0 % if pdftex
  \usepackage[T1]{fontenc}
  \usepackage[utf8]{inputenc}
\else % if luatex or xelatex
  \ifxetex
    \usepackage{mathspec}
  \else
    \usepackage{fontspec}
  \fi
  \defaultfontfeatures{Ligatures=TeX,Scale=MatchLowercase}
\fi
% use upquote if available, for straight quotes in verbatim environments
\IfFileExists{upquote.sty}{\usepackage{upquote}}{}
% use microtype if available
\IfFileExists{microtype.sty}{%
\usepackage{microtype}
\UseMicrotypeSet[protrusion]{basicmath} % disable protrusion for tt fonts
}{}
\usepackage[margin=1in]{geometry}
\usepackage{hyperref}
\hypersetup{unicode=true,
            pdftitle={MATH 447: Introduction to Stochastic Processes (Assignment 4)},
            pdfauthor={Dan Yunheum Seol},
            pdfborder={0 0 0},
            breaklinks=true}
\urlstyle{same}  % don't use monospace font for urls
\usepackage{color}
\usepackage{fancyvrb}
\newcommand{\VerbBar}{|}
\newcommand{\VERB}{\Verb[commandchars=\\\{\}]}
\DefineVerbatimEnvironment{Highlighting}{Verbatim}{commandchars=\\\{\}}
% Add ',fontsize=\small' for more characters per line
\usepackage{framed}
\definecolor{shadecolor}{RGB}{248,248,248}
\newenvironment{Shaded}{\begin{snugshade}}{\end{snugshade}}
\newcommand{\AlertTok}[1]{\textcolor[rgb]{0.94,0.16,0.16}{#1}}
\newcommand{\AnnotationTok}[1]{\textcolor[rgb]{0.56,0.35,0.01}{\textbf{\textit{#1}}}}
\newcommand{\AttributeTok}[1]{\textcolor[rgb]{0.77,0.63,0.00}{#1}}
\newcommand{\BaseNTok}[1]{\textcolor[rgb]{0.00,0.00,0.81}{#1}}
\newcommand{\BuiltInTok}[1]{#1}
\newcommand{\CharTok}[1]{\textcolor[rgb]{0.31,0.60,0.02}{#1}}
\newcommand{\CommentTok}[1]{\textcolor[rgb]{0.56,0.35,0.01}{\textit{#1}}}
\newcommand{\CommentVarTok}[1]{\textcolor[rgb]{0.56,0.35,0.01}{\textbf{\textit{#1}}}}
\newcommand{\ConstantTok}[1]{\textcolor[rgb]{0.00,0.00,0.00}{#1}}
\newcommand{\ControlFlowTok}[1]{\textcolor[rgb]{0.13,0.29,0.53}{\textbf{#1}}}
\newcommand{\DataTypeTok}[1]{\textcolor[rgb]{0.13,0.29,0.53}{#1}}
\newcommand{\DecValTok}[1]{\textcolor[rgb]{0.00,0.00,0.81}{#1}}
\newcommand{\DocumentationTok}[1]{\textcolor[rgb]{0.56,0.35,0.01}{\textbf{\textit{#1}}}}
\newcommand{\ErrorTok}[1]{\textcolor[rgb]{0.64,0.00,0.00}{\textbf{#1}}}
\newcommand{\ExtensionTok}[1]{#1}
\newcommand{\FloatTok}[1]{\textcolor[rgb]{0.00,0.00,0.81}{#1}}
\newcommand{\FunctionTok}[1]{\textcolor[rgb]{0.00,0.00,0.00}{#1}}
\newcommand{\ImportTok}[1]{#1}
\newcommand{\InformationTok}[1]{\textcolor[rgb]{0.56,0.35,0.01}{\textbf{\textit{#1}}}}
\newcommand{\KeywordTok}[1]{\textcolor[rgb]{0.13,0.29,0.53}{\textbf{#1}}}
\newcommand{\NormalTok}[1]{#1}
\newcommand{\OperatorTok}[1]{\textcolor[rgb]{0.81,0.36,0.00}{\textbf{#1}}}
\newcommand{\OtherTok}[1]{\textcolor[rgb]{0.56,0.35,0.01}{#1}}
\newcommand{\PreprocessorTok}[1]{\textcolor[rgb]{0.56,0.35,0.01}{\textit{#1}}}
\newcommand{\RegionMarkerTok}[1]{#1}
\newcommand{\SpecialCharTok}[1]{\textcolor[rgb]{0.00,0.00,0.00}{#1}}
\newcommand{\SpecialStringTok}[1]{\textcolor[rgb]{0.31,0.60,0.02}{#1}}
\newcommand{\StringTok}[1]{\textcolor[rgb]{0.31,0.60,0.02}{#1}}
\newcommand{\VariableTok}[1]{\textcolor[rgb]{0.00,0.00,0.00}{#1}}
\newcommand{\VerbatimStringTok}[1]{\textcolor[rgb]{0.31,0.60,0.02}{#1}}
\newcommand{\WarningTok}[1]{\textcolor[rgb]{0.56,0.35,0.01}{\textbf{\textit{#1}}}}
\usepackage{graphicx,grffile}
\makeatletter
\def\maxwidth{\ifdim\Gin@nat@width>\linewidth\linewidth\else\Gin@nat@width\fi}
\def\maxheight{\ifdim\Gin@nat@height>\textheight\textheight\else\Gin@nat@height\fi}
\makeatother
% Scale images if necessary, so that they will not overflow the page
% margins by default, and it is still possible to overwrite the defaults
% using explicit options in \includegraphics[width, height, ...]{}
\setkeys{Gin}{width=\maxwidth,height=\maxheight,keepaspectratio}
\IfFileExists{parskip.sty}{%
\usepackage{parskip}
}{% else
\setlength{\parindent}{0pt}
\setlength{\parskip}{6pt plus 2pt minus 1pt}
}
\setlength{\emergencystretch}{3em}  % prevent overfull lines
\providecommand{\tightlist}{%
  \setlength{\itemsep}{0pt}\setlength{\parskip}{0pt}}
\setcounter{secnumdepth}{0}
% Redefines (sub)paragraphs to behave more like sections
\ifx\paragraph\undefined\else
\let\oldparagraph\paragraph
\renewcommand{\paragraph}[1]{\oldparagraph{#1}\mbox{}}
\fi
\ifx\subparagraph\undefined\else
\let\oldsubparagraph\subparagraph
\renewcommand{\subparagraph}[1]{\oldsubparagraph{#1}\mbox{}}
\fi

%%% Use protect on footnotes to avoid problems with footnotes in titles
\let\rmarkdownfootnote\footnote%
\def\footnote{\protect\rmarkdownfootnote}

%%% Change title format to be more compact
\usepackage{titling}

% Create subtitle command for use in maketitle
\newcommand{\subtitle}[1]{
  \posttitle{
    \begin{center}\large#1\end{center}
    }
}

\setlength{\droptitle}{-2em}

  \title{MATH 447: Introduction to Stochastic Processes (Assignment 4)}
    \pretitle{\vspace{\droptitle}\centering\huge}
  \posttitle{\par}
    \author{Dan Yunheum Seol}
    \preauthor{\centering\large\emph}
  \postauthor{\par}
      \predate{\centering\large\emph}
  \postdate{\par}
    \date{April 12, 2019}

\usepackage{blkarray}

\begin{document}
\maketitle

\hypertarget{section}{%
\section{6.12}\label{section}}

\hypertarget{starting-at-noon-diners-arrive-at-a-restaurant-according-to-a-poisson-process-at-the-rate-of-five-customers-per-minute.-the-time-each-customer-spends-eating-at-the-restaurant-has-an-exponential-distribution-with-mean-40-minutes-independent-of-other-customers-and-independent-of-arrival-times.-find-the-distribution-as-well-as-the-mean-and-variance-of-the-number-of-diners-in-the-restaurant-at-2-p.m.}{%
\subsubsection{Starting at noon, diners arrive at a restaurant according
to a Poisson process at the rate of five customers per minute. The time
each customer spends eating at the restaurant has an exponential
distribution with mean 40 minutes, independent of other customers and
independent of arrival times. Find the distribution, as well as the mean
and variance, of the number of diners in the restaurant at 2
p.m.}\label{starting-at-noon-diners-arrive-at-a-restaurant-according-to-a-poisson-process-at-the-rate-of-five-customers-per-minute.-the-time-each-customer-spends-eating-at-the-restaurant-has-an-exponential-distribution-with-mean-40-minutes-independent-of-other-customers-and-independent-of-arrival-times.-find-the-distribution-as-well-as-the-mean-and-variance-of-the-number-of-diners-in-the-restaurant-at-2-p.m.}}

Let \[
N_t := \text{The number of customers arriving at diner on time t}
\] We have \[
(N_t)_{t \geq 0} \sim PP(5)
\] and define the time spent \[
T_k := \text{time customer k spends} \sim Exp(\frac{1}{40})
\] where \$T\_k \perp !!! \perp T\_j \$ if \(j \neq k\) and
\(T_k \perp \!\!\! \perp N_k\)

Then define\\
\[
C_t := \text{The number of customers at the diner on time t} 
\]

Firstly, you condition on \(N_t\)

\[
P(C_t = k) = \sum_{n=k}^\infty P(C_t = k | N_t = n)P(N_t = n)
\] By the definition of Poisson process, \[
N_t \sim Pois(5t) \implies (N_t = n) = \frac{e^{-5t}(5t)^n}{n!} 
\] We obtain \[
P(C_t = k) = \sum_{n=k}^\infty P(C_t = k | N_t = n)\frac{e^{-5t}(5t)^n}{n!}
\]

Secondly, we find the probability for \(C_t | N_t\).

Let \(S_i\), be the uniformly distributed arrival times conditioned on
\(N_t = n\), thus we have

\[
P(C_t = k | N_t = n) = P(k \text{ of } S_i + T_i > t) = P(k \text{ of } S_i + T_i > t)  = P(k \text{ of } U_{(i)} + T_i > t)  = 
\]

\[
P(k \text{ of } U_i + T_i > t)  = {n\choose k} p^k (1-p)^{n-k}  =
\] where

\[
p =  P(U_i + T_i > t) = \frac{1}{t}\int_{0}^t P(T_i > t - u) du = \frac{1}{t}\int_{0}^t P( T_i > t-u |  u \leq t) du = \frac{1}{t}\int_{0}^t P(T_i > x) dx =
\]

\[
= 
\frac{1}{t} \int_{0}^t 1 - exp(-\frac{1}{40}x) dx = \frac{1}{t} \{ t +40exp(-\frac{1}{40}x) \bigg |^t _0\}
\] then we have \[
P(C_t = k) = \sum_{n=k}^\infty{n\choose k} p^k (1-p)^{n-k}  \frac{e^{-5t}(5t)^n}{n!} = \frac{n!}{k!(n-k)!} p^k (1-p)^{n-k}  \frac{e^{-5t}(5t)^n}{n!} =
\]

\[
\sum_{n=k}^\infty \frac{p^k(5t)^k (1-p)^{n-k} (5t)^{n-k}}{k!(n-k)!} e^{-5pt} e^{-5(1-p)t} =
\]

\[
e^{-5pt}\frac{(5pt)^k}{k!} e^{-5(1-p)t} \sum_{n=k}^\infty \frac{(5(1-p)t)^{m-k}}{(n-k)!} = 
\]

but \[
\sum_{n=k}^\infty \frac{(5(1-p)t)^{m-k}}{(n-k)!} = e^{5(1-p)t}
\] so it follows that \[
P(C_t = k) = e^{-5pt}\frac{(5pt)^k}{k!}
\] So \((C_t)_{t \geq 0}\) is a Poisson process with
\(\lambda = 5p\)\textbackslash{}

We would have \[
E[C_t] = Var[C_t] = 5pt
\] and let's calculate for when t = 120

we will have

\[
5pt= 5* \frac{120}{120}40(1-e^{-3})=190
\]

\#6.15

\hypertarget{failures-occur-for-a-mechanical-process-according-to-a-poisson-process.-failures-are-classiied-as-either-major-or-minor.-major-failures-occur-at-the-rate-of-1.5-failures-per-hour.-minor-failures-occur-at-the-rate-of-3.0-failures-per-hour.}{%
\subsubsection{Failures occur for a mechanical process according to a
Poisson process. Failures are classiied as either major or minor. Major
failures occur at the rate of 1.5 failures per hour. Minor failures
occur at the rate of 3.0 failures per
hour.}\label{failures-occur-for-a-mechanical-process-according-to-a-poisson-process.-failures-are-classiied-as-either-major-or-minor.-major-failures-occur-at-the-rate-of-1.5-failures-per-hour.-minor-failures-occur-at-the-rate-of-3.0-failures-per-hour.}}

\[
F_t =  F_t^{(M)} + F_t^{(m)}
\] where \[
\begin{array}{c}
F_t^{(M)} \sim PP(1.5)\ & F_t ^{(m)} \sim PP(3)
\end{array}
\] Remark that we have \[
F_t \sim PP(4.5)
\]

\hypertarget{a-find-the-probability-that-two-failures-occur-in-1-hour.}{%
\subsubsection{(a) Find the probability that two failures occur in 1
hour.}\label{a-find-the-probability-that-two-failures-occur-in-1-hour.}}

\[
P(F_1 = 2) = e^{-4.5}\frac{(4.5)^2}{2!} = 0.1125
\]

\hypertarget{b-find-the-probability-that-in-half-an-hour-no-major-failures-occur.}{%
\subsubsection{(b) Find the probability that in half an hour, no major
failures
occur.}\label{b-find-the-probability-that-in-half-an-hour-no-major-failures-occur.}}

\[
P(F_{0.5}^{(M)}= 0) = e^{-1.5 * 0.5} = e^{(-0.75)} = 0.4724
\]

\hypertarget{c-find-the-probability-that-in-2-hours-at-least-two-major-failures-occur-or-at-least-two-minor-failures-occur.}{%
\subsubsection{(c) Find the probability that in 2 hours, at least two
major failures occur or at least two minor failures
occur.}\label{c-find-the-probability-that-in-2-hours-at-least-two-major-failures-occur-or-at-least-two-minor-failures-occur.}}

Since major failures and minor failures happen independently, \[
P(F_2^{(M)} \geq 2 \lor F_2^{(m)} \geq 2) = P(F_2^{(M)} \geq 2) + P(F_2^{(m)} \geq 2) - P(F_2^{(M)} \geq 2 ) P( F_2^{(m)} \geq 2) =
\]

\[
(1-P(F_2^{(M)} \leq 1)  + (1-P(F_2^{(m)} \leq 1)  -  (1-P(F_2^{(M)} \leq 1) (1-P(F_2^{(m)} \leq 1) =
\]

\[
1 - P(F_2^{(M)} \leq 1)-P(F_2^{(m)} \leq 1) + P(F_2^{(M)} \leq 1 + P(F_2^{(m)} \leq 1) - P(F_2^{(M)} \leq 1)P(F_2^{(m)} \leq 1) =
\]

\[
1 - P(F_2^{(M)} \leq 1)P(F_2^{(m)} \leq 1) = 1 - (e^{-3}\sum_{i=0}^1 3^i )(e^{-6} \sum_{j=0}^1 6^i) = 0.9965
\] \# 6.33

\hypertarget{let-s_1-s_2-...-be-the-arrival-times-of-a-poisson-process-with-parameter-lambda.-given-the-time-of-the-nth-arrival-ind-the-expected-time-es_1s_n-of-the-first-arrival.}{%
\subsubsection{\texorpdfstring{Let \(S_1, S_2, ...\) be the arrival
times of a Poisson process with parameter \(\lambda\). Given the time of
the \(n^{th}\) arrival, ind the expected time \(E(S_1|S_n)\) of the
first
arrival.}{Let S\_1, S\_2, ... be the arrival times of a Poisson process with parameter \textbackslash{}lambda. Given the time of the n\^{}\{th\} arrival, ind the expected time E(S\_1\textbar{}S\_n) of the first arrival.}}\label{let-s_1-s_2-...-be-the-arrival-times-of-a-poisson-process-with-parameter-lambda.-given-the-time-of-the-nth-arrival-ind-the-expected-time-es_1s_n-of-the-first-arrival.}}

\[
E[S_1|S_n] = E[S_1 | N_{S_n} =n] = E[U_{(1)} | N_{S_n} = n]
\]

where \(U_i \sim Uniform(0, S_n)\)

Remark if \(S_n = t\) \[
P(U_{(1)} \leq u | N_{S_n} =n) = 1-P(U_{(1)} > u | N_{S_n} =n) = 1- (1-P(U \leq u))^{n-1} = 1 - (1 - \frac{u}{t})^{n-1}
\]

\[
\implies  f_{U_{(1)}}(u) =  \frac{(n-1)}{t}(1 - \frac{u}{t})^{n-2}
\] So \[
E[U_{(1)} \leq u | S_n = t)] = \int_{0}^t \frac{(n-1)u}{t}(1 - \frac{u}{t})^{n-2} du 
\] If you parametrize \(\frac{u}{t} = x\) you have
\(\frac{1}{t} du = dx\) so we have \(du = t dx\)

It follows that

\[
E[U_{(1)} \leq u | S_n = t)] = (n-1)t \int_{0}^1 x (1-x)^{n-2} dx = (n-1) t \frac{\Gamma(2)\Gamma(n-1)}{\Gamma(n+1)} = \frac{t}{n} = \frac{S_n}{n}
\]

\hypertarget{section-1}{%
\section{7.24}\label{section-1}}

\hypertarget{customers-arrive-at-a-busy-food-truck-according-to-a-poisson-process-with-parameter-lambda.-if-there-are-i-people-already-in-line-the-customer-will-join-the-line-with-probability-1i-1.-assume-that-the-chef-at-the-truck-takes-on-average-alpha-minutes-to-process-an-order.}{%
\subsubsection{\texorpdfstring{Customers arrive at a busy food truck
according to a Poisson process with parameter \(\lambda\). If there are
\(i\) people already in line, the customer will join the line with
probability \(1/(i + 1)\). Assume that the chef at the truck takes, on
average, \(\alpha\) minutes to process an
order.}{Customers arrive at a busy food truck according to a Poisson process with parameter \textbackslash{}lambda. If there are i people already in line, the customer will join the line with probability 1/(i + 1). Assume that the chef at the truck takes, on average, \textbackslash{}alpha minutes to process an order.}}\label{customers-arrive-at-a-busy-food-truck-according-to-a-poisson-process-with-parameter-lambda.-if-there-are-i-people-already-in-line-the-customer-will-join-the-line-with-probability-1i-1.-assume-that-the-chef-at-the-truck-takes-on-average-alpha-minutes-to-process-an-order.}}

Define the number of people as \((X_t)_{t \geq 0}\)

Remark the time that it takes to process an order will be exponentially
distributed with parameter \(1/ \alpha\)

The number of people in line is a birth-death process with
\(\lambda_i = \frac{\lambda}{i+1}\) and
\(\mu_i = \mu = \frac{1}{\alpha}\)

We will find the stationary distribution first, remarking
\$\frac{\lambda_i}{\mu_i} = \frac{(\lambda \alpha)}{i+1} \$ \[
\begin{array}{c}
\pi_1 = \lambda\alpha \pi_0 \\
\pi_2 = \frac{\lambda\alpha}{2} \pi_1 = \frac{(\lambda\alpha)^2}{2!} \pi_0 \\
... \\
\pi_k = \frac{(\lambda\alpha)^k}{k!} \pi_0
\end{array}
\]

\[
\pi_0 = \bigg( \sum_{k=0}^\infty \frac{(\lambda\alpha)^k}{k!} \bigg)^{-1} = e^{-\lambda \alpha}
\] So long-term number of people in line is a poisson variable. \#\#\#
(a) Find the long-term average number of people in line.

Since the long term number of people is a Poisson random variable, the
expected number would be\\
\[\lambda \alpha\]

\hypertarget{b-find-the-long-term-probability-that-there-are-at-least-two-people-in-line.}{%
\subsubsection{(b) Find the long-term probability that there are at
least two people in
line.}\label{b-find-the-long-term-probability-that-there-are-at-least-two-people-in-line.}}

\[
\lim_{t \rightarrow \infty } P(X_t \geq 2) = 1- (\pi_0 + \pi_1) = 1 - e^{-\lambda \alpha}(1+ \lambda \alpha)
\]

\hypertarget{section-2}{%
\section{7.27}\label{section-2}}

\hypertarget{recall-the-discrete-time-ehrenfest-doglea-model-of-example-3.7.-in-the-continuous-time-version-there-are-n-fleas-distributed-between-two-dogs.-fleas-jump-from-one-dog-to-another-independently-at-rate-lambda.-let-x_t-denote-the-number-of-fleas-on-the-first-dog.}{%
\subsubsection{\texorpdfstring{Recall the discrete-time Ehrenfest
dog--lea model of Example 3.7. In the continuous-time version, there are
\(N\) fleas distributed between two dogs. Fleas jump from one dog to
another independently at rate \(\lambda\). Let \(X_t\) denote the number
of fleas on the first
dog.}{Recall the discrete-time Ehrenfest dog--lea model of Example 3.7. In the continuous-time version, there are N fleas distributed between two dogs. Fleas jump from one dog to another independently at rate \textbackslash{}lambda. Let X\_t denote the number of fleas on the first dog.}}\label{recall-the-discrete-time-ehrenfest-doglea-model-of-example-3.7.-in-the-continuous-time-version-there-are-n-fleas-distributed-between-two-dogs.-fleas-jump-from-one-dog-to-another-independently-at-rate-lambda.-let-x_t-denote-the-number-of-fleas-on-the-first-dog.}}

\hypertarget{a-show-that-the-process-is-a-birth-and-death-process.-give-the-birth-and-death-rates.}{%
\subsubsection{(a) Show that the process is a birth-and-death process.
Give the birth and death
rates.}\label{a-show-that-the-process-is-a-birth-and-death-process.-give-the-birth-and-death-rates.}}

Let \(Y_{i1} \sim exp(\lambda)\) be the time where ith flea jumps from
the second dog to the first dog. Let \(Y_{i2} \sim exp(\lambda)\) be the
time where ith flea jumps from the first dog to the second dog.

Suppose \(X_t = i\), then the number of fleas on the first dog increases
by one the first time that one of the fleas on the second dog (there are
\(N-X_t = N-i\) fleas on it) jumps. The time index where the first jump
from the second dog happens at the minimum of the independent
exponential random variables, which is also an exponential random
variable with the parameter being the sum of it.

\[
min(Y_{11},... Y_{(N-i)1}) \sim exp((N-i) \lambda ) = q_{i, i+1}
\]

\[
min(Y_{11},... Y_{(i)1}) \sim exp(i\lambda) = q_{i, i-1}
\]

\hypertarget{b-find-the-stationary-distribution.}{%
\subsubsection{(b) Find the stationary
distribution.}\label{b-find-the-stationary-distribution.}}

The detailed balance question would be \[
\pi_i (N-i)\lambda = \pi_{i+1} (i+1) \lambda
\]

\[
\begin{array}{c}
\pi_0 N \lambda = \pi_1 \lambda \implies \pi_1 = N \pi_0  \\
\pi_1 (N-1) \lambda = \pi_2 2\lambda \implies \pi_2 = \frac{N-1}{2} \pi_1 =  \frac{N(N-1)}{2} \pi_0 \\
...\\
\pi_k (N-k) \lambda = \pi_{k+1} (k+1)\lambda \implies \pi_{k+1} = {N\choose k} \pi_0  \\
\implies 1 = \pi_0 \sum_{k=0}^N {N\choose k}  \implies 
\\ \pi_0 = \frac{1}{\sum_{k=0}^N {N\choose k} } = \frac{1}{2^N}
\end{array}
\]

So we can find our stationary distribution \[
\pi =(\pi_0, \pi_1, ... \pi_N)
\] where \(k \in \{0, 1, ..., N\}\) \[
\pi_k = {N \choose k} \frac{1}{2^N}
\] which is a binomial distribution with parameters \(N, 1/2\)

\hypertarget{c-assume-that-leas-jump-at-the-rate-of-2-per-minute.-if-there-are-10-fleas-on-cooper-and-no-fleas-on-lisa-how-long-on-average-will-it-take-for-lisa-to-get-4-fleas}{%
\subsubsection{(c) Assume that leas jump at the rate of 2 per minute. If
there are 10 fleas on Cooper and no fleas on Lisa, how long, on average,
will it take for Lisa to get 4
fleas?}\label{c-assume-that-leas-jump-at-the-rate-of-2-per-minute.-if-there-are-10-fleas-on-cooper-and-no-fleas-on-lisa-how-long-on-average-will-it-take-for-lisa-to-get-4-fleas}}

Let \(X_t\) be the number of fleas on Lisa. We set 4 as an absorbing
state. We would not counting the cases where it reaches 4 since we are
starting from 0.

So we construct a matrix in canonical form, which looks like this:

\[
Q_{canonical}=\left[\begin{array}{ccccc} 0 &0&0&0&0\\ 0&-20&20&0&0 \\ 0 & 2 & -20 & 18 &0\\0&0&4&-20&16\\14&0&0&6&-20 \end{array} \right]
\] with V \[
\left[\begin{array}{cccc} -20&20&0&0 \\ 2 & -20 & 18 &0\\0&4&-20&16\\0&0&6&-20 \end{array} \right]
\]

and we calculate \(F =-V^{-1}\) as below

\begin{Shaded}
\begin{Highlighting}[]
\NormalTok{labels <-}\StringTok{ }\KeywordTok{c}\NormalTok{(}\DecValTok{4}\NormalTok{,}\DecValTok{0}\NormalTok{,}\DecValTok{1}\NormalTok{,}\DecValTok{2}\NormalTok{,}\DecValTok{3}\NormalTok{)}

\NormalTok{row1 <-}\StringTok{ }\KeywordTok{rep}\NormalTok{(}\DecValTok{0}\NormalTok{, }\DecValTok{5}\NormalTok{)}
\NormalTok{row2 <-}\StringTok{ }\KeywordTok{c}\NormalTok{(}\DecValTok{0}\NormalTok{,}\OperatorTok{-}\DecValTok{20}\NormalTok{,}\DecValTok{20}\NormalTok{,}\DecValTok{0}\NormalTok{,}\DecValTok{0}\NormalTok{)}
\NormalTok{row3 <-}\StringTok{ }\KeywordTok{c}\NormalTok{(}\DecValTok{0}\NormalTok{,}\DecValTok{2}\NormalTok{,}\OperatorTok{-}\DecValTok{20}\NormalTok{,}\DecValTok{18}\NormalTok{,}\DecValTok{0}\NormalTok{)}
\NormalTok{row4 <-}\StringTok{ }\KeywordTok{c}\NormalTok{(}\DecValTok{0}\NormalTok{,}\DecValTok{0}\NormalTok{,}\DecValTok{4}\NormalTok{,}\OperatorTok{-}\DecValTok{20}\NormalTok{,}\DecValTok{16}\NormalTok{)}
\NormalTok{row5 <-}\StringTok{ }\KeywordTok{c}\NormalTok{(}\DecValTok{14}\NormalTok{,}\DecValTok{0}\NormalTok{,}\DecValTok{0}\NormalTok{,}\DecValTok{6}\NormalTok{,}\OperatorTok{-}\DecValTok{20}\NormalTok{)}

\NormalTok{Q <-}\StringTok{ }\KeywordTok{rbind}\NormalTok{(row1,row2,row3,row4,row5)}

\KeywordTok{rownames}\NormalTok{(Q) <-}\StringTok{ }\NormalTok{labels}
\KeywordTok{colnames}\NormalTok{(Q) <-}\StringTok{ }\NormalTok{labels}
\KeywordTok{print}\NormalTok{(}\StringTok{"Q"}\NormalTok{)}
\end{Highlighting}
\end{Shaded}

\begin{verbatim}
## [1] "Q"
\end{verbatim}

\begin{Shaded}
\begin{Highlighting}[]
\NormalTok{Q}
\end{Highlighting}
\end{Shaded}

\begin{verbatim}
##    4   0   1   2   3
## 4  0   0   0   0   0
## 0  0 -20  20   0   0
## 1  0   2 -20  18   0
## 2  0   0   4 -20  16
## 3 14   0   0   6 -20
\end{verbatim}

\begin{Shaded}
\begin{Highlighting}[]
\NormalTok{V <-}\StringTok{ }\NormalTok{Q[}\KeywordTok{c}\NormalTok{(}\DecValTok{2}\NormalTok{,}\DecValTok{3}\NormalTok{,}\DecValTok{4}\NormalTok{,}\DecValTok{5}\NormalTok{), }\KeywordTok{c}\NormalTok{(}\DecValTok{2}\NormalTok{,}\DecValTok{3}\NormalTok{,}\DecValTok{4}\NormalTok{,}\DecValTok{5}\NormalTok{)]}
\KeywordTok{print}\NormalTok{(}\StringTok{"V"}\NormalTok{)}
\end{Highlighting}
\end{Shaded}

\begin{verbatim}
## [1] "V"
\end{verbatim}

\begin{Shaded}
\begin{Highlighting}[]
\NormalTok{V}
\end{Highlighting}
\end{Shaded}

\begin{verbatim}
##     0   1   2   3
## 0 -20  20   0   0
## 1   2 -20  18   0
## 2   0   4 -20  16
## 3   0   0   6 -20
\end{verbatim}

\begin{Shaded}
\begin{Highlighting}[]
\NormalTok{f <-}\StringTok{ }\OperatorTok{-}\KeywordTok{solve}\NormalTok{(V)}
\KeywordTok{print}\NormalTok{(}\StringTok{"F"}\NormalTok{)}
\end{Highlighting}
\end{Shaded}

\begin{verbatim}
## [1] "F"
\end{verbatim}

\begin{Shaded}
\begin{Highlighting}[]
\NormalTok{f}
\end{Highlighting}
\end{Shaded}

\begin{verbatim}
##              0           1          2          3
## 0 0.0575396825 0.075396825 0.08928571 0.07142857
## 1 0.0075396825 0.075396825 0.08928571 0.07142857
## 2 0.0019841270 0.019841270 0.08928571 0.07142857
## 3 0.0005952381 0.005952381 0.02678571 0.07142857
\end{verbatim}

and we have our expected time as the sum of entries of the first row:

\begin{Shaded}
\begin{Highlighting}[]
\KeywordTok{sum}\NormalTok{(f[}\DecValTok{1}\NormalTok{,}\KeywordTok{c}\NormalTok{(}\DecValTok{1}\OperatorTok{:}\DecValTok{4}\NormalTok{)])}
\end{Highlighting}
\end{Shaded}

\begin{verbatim}
## [1] 0.2936508
\end{verbatim}

\(0.294\) minutes will be our waiting time. (For some reason some of my
friends kept saying the answer would be 0.45 minutes; why?)


\end{document}
